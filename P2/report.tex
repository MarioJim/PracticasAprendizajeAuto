\documentclass[sigconf,authorversion,nonacm]{acmart}

\usepackage{listings}

\AtBeginDocument{%
  \providecommand\BibTeX{{%
    \normalfont B\kern-0.5em{\scshape i\kern-0.25em b}\kern-0.8em\TeX}}}

\begin{document}

\title{Práctica 2 \\ Regresión logística}

\author{Mario Emilio Jiménez Vizcaíno}
\email{A01173359@itesm.mx}
\affiliation{%
  \institution{Tecnológico de Monterrey \\ Ingeniería en Tecnologías Computacionales}
  \city{Monterrey, N.L.}
  \country{México}
}

\author{Jesus Abraham Haros Madrid}
\email{A01252642@itesm.mx}
\affiliation{%
  \institution{Tecnológico de Monterrey \\ Ingeniería en Tecnologías Computacionales}
  \city{Monterrey, N.L.}
  \country{México}
}


\begin{abstract}
\end{abstract}

\maketitle

\section{Introducción}


\section{Conceptos previos}
\begin{itemize}
  \item Concepto
\end{itemize}


\section{Metodología}

\subsection{Regresión logística con \textit{sklearn}}

\subsubsection{Dataset DEFAULT}\hfill\\

\subsubsection{Dataset GENERO}\hfill\\

\subsection{Regresión logística con Gradiente Descendente}

\subsubsection{Dataset DEFAULT}\hfill\\

\subsubsection{Dataset GENERO}\hfill\\


\section{Resultados}

\subsection{Regresión logística con \textit{sklearn}}

\subsubsection{Dataset DEFAULT}\hfill\\

\subsubsection{Dataset GENERO}\hfill\\

\subsection{Regresión logística con Gradiente Descendente}

\subsubsection{Dataset DEFAULT}\hfill\\

\subsubsection{Dataset GENERO}\hfill\\


\section{Conclusiones y reflexiones}

\subsection{Refrexión de Abraham}

\subsection{Reflexión de Mario}

\bibliographystyle{ACM-Reference-Format}
\bibliography{references}

\clearpage

\appendix

\lstdefinestyle{customstyle}{
  frame=single,
  numbers=left,
  numbersep=5pt,
  showstringspaces=false
}
\lstset{style=customstyle}

\begin{figure*}
  \section{Código de regresión logística con sklearn del dataset DEFAULT}
  \lstinputlisting[language=Python]{default_sklearn.py}
\end{figure*}

\begin{figure*}
  \section{Código de regresión logística con sklearn del dataset GENERO}
  \lstinputlisting[language=Python]{genero_sklearn.py}
\end{figure*}

\begin{figure*}
  \section{Nuestra implementación de gradiente descendente}
  \lstinputlisting[language=Python]{GradientDescent.py}
\end{figure*}

\begin{figure*}
  \section{Código de regresión logística con gradiente descendente del dataset DEFAULT}
  \lstinputlisting[language=Python]{default_gradiente.py}
\end{figure*}

\begin{figure*}
  \section{Código de regresión logística con gradiente descendente del dataset GENERO}
  \lstinputlisting[language=Python]{genero_gradiente.py}
\end{figure*}

\begin{figure*}
  \section{Código de generación de gráficas}
  \lstinputlisting[language=Python]{graphs.py}
\end{figure*}

\end{document}
\endinput
