\documentclass[sigconf,authorversion,nonacm]{acmart}

\usepackage{listings}

\AtBeginDocument{%
  \providecommand\BibTeX{{%
    \normalfont B\kern-0.5em{\scshape i\kern-0.25em b}\kern-0.8em\TeX}}}

\begin{document}

\title{Práctica 3 \\ Clasificadores k-NN y regresión logística}

\author{Mario Emilio Jiménez Vizcaíno}
\email{A01173359@itesm.mx}
\affiliation{%
  \institution{Tecnológico de Monterrey \\ Ingeniería en Tecnologías Computacionales}
  \city{Monterrey, N.L.}
  \country{México}
}

\author{Jesus Abraham Haros Madrid}
\email{A01252642@itesm.mx}
\affiliation{%
  \institution{Tecnológico de Monterrey \\ Ingeniería en Tecnologías Computacionales}
  \city{Monterrey, N.L.}
  \country{México}
}


\begin{abstract}
\textbf{TODO}
\end{abstract}

\maketitle

\section{Introducción}
\textbf{TODO}

\section{Conceptos previos}
\begin{itemize}
  \item Programación básica en los lenguajes R y Python
  \item Conocimiento de las librerías scikit-learn, pandas y numpy
  \item Conocimientos de estadística y de regresión logística
\end{itemize}


\section{Metodología}

\subsection{Datasets}
Para comparar ambas implementaciones utilizamos dos datasets, expuestos a continuación:

\subsubsection{Dataset DEFAULT}\hfill\\
Este dataset está compuesto por 10,000 filas, cada una representa un cliente de un banco que puede o no cumplir con los pagos de su tarjeta de crédito (columna "default"). De cada cliente tenemos la siguiente información:
\begin{itemize}
  \item Columna "default": Tiene los valores "Yes"/"No", representa si la persona realizó el pago mínimo a su tarjeta de crédito.
  \item Columna "student": Valores "Yes"/"No", representa si el cliente es un estudiante en ese momento.
  \item Columna "balance": Número decimal positivo que representa el balance de la tarjeta de crédito del cliente. Promedio de 835.4, números en el rango [0, 2654.3].
  \item Columna "income": Número decimal positivo que representa los ingresos que tiene el cliente. Promedio de 33517, números en el rango [772, 73554]
\end{itemize}

De este dataset, nuestro objetivo es predecir la columna "default" a partir de los otros tres parámetros, y como preparación cambiamos los valores de "student" ("Yes"/"No") a valores 1 y 0 respectivamente.


\subsubsection{Dataset GENERO}\hfill\\
Este dataset representa las mediciones de peso y altura de 10,000 personas, en conjunto con el género de la persona a la que se realizaron las medidas. Las columnas son:
\begin{itemize}
  \item "Gender": Valores "Male"/"Female", el género de la persona a la que le corresponde esta fila de mediciones.
  \item "Height": La altura de la persona en pulgadas, promedio de 66.37, en el rango [54.26 y 79.00].
  \item "Weight": El peso de la persona en libras, promedio de 161.4, en el rango [64.7, 270.0].
\end{itemize}

La columna objetivo seleccionada de este dataset fue el género ya que tiene dos clasificaciones.

\subsection{Clasificación con k-NN}

\subsubsection{Dataset DEFAULT}\hfill\\
\textbf{TODO}

El código fuente de este ejemplo se encuentra en el apéndice C.

\subsubsection{Dataset GENERO}\hfill\\
\textbf{TODO}

El código fuente puede ser encontrado en el apéndice D.

\subsection{Regresión logística}
Para la primera parte de la práctica, en la que utilizamos la implementación de \textit{sci-kit learn}, seleccionamos la clase \newline\textit{sklearn.linear\_model.LogisticRegression}\cite{scikit-learn} para nuestros scripts.

\subsubsection{Dataset DEFAULT}\hfill\\
Para este dataset primero leemos el archivo CSV a un \textit{Dataframe} de \textit{pandas}, transformamos las columnas "default" y "student" para que contengan valores booleanos y enteros respectivamente, seleccionamos las columnas que nos servirán como variables independientes (columnas "student", "balance" e "income") y variable dependiente (columna "default"). Después partimos las filas del dataset en una porción del 80\% que usaremos para entrenar el modelo, y otra porción del 20\% para probarlo.

Instanciamos el modelo de regresión de \textit{sklearn}, lo entrenamos con los datos y después predecimos la variable dependiente con el modelo para así compararlo con los datos reales de prueba, usando una medida de tasa de precisión y la matriz de confusión.

El código de este ejemplo puede se encuentra en el apéndice A.

\subsubsection{Dataset GENERO}\hfill\\
Para este dataset realizamos un procedimiento similar: leer el dataset para crear un \textit{Dataframe}, seleccionar las columnas de variables independientes ("Height" y "Weight") y la dependiente ("Gender"), dividir el dataset en 80\%/20\%, entrenar el modelo, y predecir la variable dependiente para los datos de prueba, para así comparar estos con los datos reales.

El código para esta sección se encuentra en el apéndice B.


\section{Resultados}

\subsection{Clasificación con k-NN}

\subsubsection{Dataset DEFAULT}\hfill\\
\textbf{TODO}

\subsubsection{Dataset GENERO}\hfill\\
\textbf{TODO}

\subsection{Regresión logística}

\subsubsection{Dataset DEFAULT}\hfill\\
\textbf{TODO}

\subsubsection{Dataset GENERO}\hfill\\
\textbf{TODO}


\section{Conclusiones y reflexiones}
\textbf{TODO}

\subsection{Refrexión de Abraham}
\textbf{TODO}

\subsection{Reflexión de Mario}
\textbf{TODO}


\bibliographystyle{ACM-Reference-Format}
\bibliography{references}

\clearpage

\appendix

\lstdefinestyle{customstyle}{
  frame=single,
  numbers=left,
  numbersep=5pt,
  showstringspaces=false
}
\lstset{style=customstyle}

\begin{figure*}
  \section{Código de regresión logística del dataset DEFAULT}
  % \lstinputlisting[language=Python]{default_sklearn.py}
\end{figure*}

\begin{figure*}
  \section{Código de regresión logística del dataset GENERO}
  % \lstinputlisting[language=Python]{genero_sklearn.py}
\end{figure*}

\begin{figure*}
  \section{Código de clasificación k-NN del dataset DEFAULT}
  % \lstinputlisting[language=Python]{default_gradiente.py}
\end{figure*}

\begin{figure*}
  \section{Código de clasificación k-NN del dataset GENERO}
  % \lstinputlisting[language=Python]{genero_gradiente.py}
\end{figure*}

\begin{figure*}
  \section{Código de generación de gráficas}
  % \lstinputlisting[language=Python]{graphs.py}
\end{figure*}

\end{document}
\endinput
